\documentclass[a4paper,12pt]{extarticle}

\usepackage[top=2cm, bottom=2cm, left=3cm, right=1cm]{geometry}
\usepackage{indentfirst}
\usepackage[T1, T2A]{fontenc}
\usepackage[utf8]{inputenc}
\usepackage[english, russian]{babel}
\usepackage{latexsym,amsthm,amsmath,amssymb,amsxtra,stmaryrd}
\usepackage{amsmath}
\usepackage{amsfonts}
\usepackage{mathrsfs}
\usepackage{xcolor}
\usepackage{tempora}
\usepackage{graphicx,multicol}
\usepackage{url}
\usepackage[russian]{babel}
\usepackage{indentfirst} %отступ в первом абзаце
\usepackage{hhline}
\usepackage{icomma} %запятая как точка без пробела в формулах
\usepackage{tikz}
\usepackage{graphicx}
\usepackage{subcaption}
\usepackage{mwe}
\usepackage{tabularx}
\usepackage{float}
\usepackage{tikzpeople}
\usepackage{fontawesome5}
\usepackage{hyperref}
\usepackage{fancyhdr} % для колонтитулов
\usepackage{lipsum}

\newcommand\tab[1][0.5cm]{\hspace*{#1}}
\newcommand\doubletab[1][1.3cm]{\hspace*{#1}}
\newcommand\tripletab[1][2.4cm]{\hspace*{#1}}
\newcommand*\xor{\mathbin{\oplus}}
\newtheorem{theorem}{Теорема}[section]

\newtheorem{df}[theorem]{Определение}
\newtheorem{teo}[theorem]{Теорема}
\newtheorem{lem}[theorem]{Лемма}
\newtheorem{prp}[theorem]{Предложение}
\newtheorem*{hyp}{Предположение}
\newtheorem{ass}[theorem]{Утверждение}
\newtheorem{cor}[theorem]{Следствие}
\newtheorem*{ev}{Доказательство}
\newtheorem{ntc}[theorem]{Замечание}


\newcommand{\al}{{\alpha}}
\newcommand{\bet}{{\beta}}
\newcommand{\de}{{\delta}}
\newcommand{\De}{{\Delta}}
\newcommand{\om}{{\omega}}
\newcommand{\ph}{{\varphi}}
\newcommand{\ep}{{\varepsilon}}
\newcommand{\FF}{\mathbb{F}}
\newcommand{\ZZ}{\mathbb{Z}}
\newcommand{\RR}{\mathbb{R}}
\newcommand{\NN}{\mathbb{N}}
\newcommand{\QQ}{\mathbb{Q}}
\newcommand{\bx}{{\bf x}}
\newcommand{\na}{{\nabla}}
\newcommand{\Hom}{\mathop{\rm Hom}\nolimits}
\newcommand{\CC}{\mathbb{C}}
\newcommand{\cC}{{\cal C}}
\newcommand{\tr}{\mathop{\rm tr}\nolimits}
\newcommand{\xp}{\boxplus}

\newcommand{\cB}{{\cal B}}
\newcommand{\cA}{{\cal A}}
\newcommand{\cH}{{\cal H}}
\newcommand{\cM}{{\cal M}}
\newcommand{\cT}{{\cal T}}
\newcommand{\cK}{{\cal K}}
\newcommand{\cP}{{\cal P}}
\newcommand{\cQH}{{\cal QH}}
\newcommand{\cMH}{{\cal MH}}
\newcommand{\cMHk}{{{\cal MH}_k}}

\newcommand{\TT}{\mathbb{T}}

\newcommand{\ba}{{\bf a}}
\newcommand{\bo}{{\bf 0}}
\newcommand{\bb}{{\bf b}}
\newcommand{\bL}{{\bf L}}

\newcommand{\mto}{\mapsto}

\newcommand{\ov}{\overline}
\newcommand{\un}{\underline}

\newcommand{\op}{{\oplus}}

\newcommand{\ovp}{\overline{p}}
\newcommand{\unp}{\underline{p}}

\newcommand{\setz}{\setminus\{0\}}
\newcommand{\til}{\widetilde}

\newcommand{\cF}{{\cal F}}

\newcommand*{\EN}{\mathsf{M}}
\newcommand*{\D}{\mathsf{D}}
\newcommand*{\cov}{\mathsf{cov}}
\newcommand*{\corr}{\mathsf{corr}}
\newcommand*{\prob}{\mathsf{P}}
\newcommand{\Ind}{\mathop{\rm Ind}\nolimits}
\DeclareMathOperator{\sign}{sign}
\newcommand{\eqdef}{\stackrel{\mathrm{def}}{=}}
\newcommand*{\HH}[1]{\mathcal{H}\left( #1\right) }
\newcommand*{\Int}{\int\limits}
\newcommand*{\SUM}{\sum\limits}
\newcommand*{\Exp}[1]{\exp\left\{ #1 \right\}}
\renewcommand*{\Pr}[1]{\prob\left( #1 \right)}

\usepackage{tabularx}
\newcolumntype{L}[1]{>{\hsize=#1\hsize\raggedright\arraybackslash}X}%
\newcolumntype{R}[1]{>{\hsize=#1\hsize\raggedleft\arraybackslash}X}%
\newcolumntype{C}[1]{>{\hsize=#1\hsize\centering\arraybackslash}X}%
\newcolumntype{J}[1]{>{\hsize=#1\hsize\noindent\justifying\arraybackslash}X}%

\newcommand{\T}{\rule{0pt}{2.3ex}} %top strut
\newcommand{\B}{\rule[-2ex]{0pt}{0ex}} %bottom strut
\usepackage{fourier-orns}

\title{Пороговые подписи} 
\author{ Юрченко В.А.$^{1}$\\
	$^{1}$ МФТИ ФПМИ,	\\	
}

% \date{\today}
% \pagestyle{myheadings}
% \pagestyle{fancy}
% \fancyhf{}
% \renewcommand{\headrule}{
% 	\vspace{-8pt}\hrulefill
% 	\raisebox{-2.1pt}{\quad\decofourleft\decotwo\decofourright\quad}\hrulefill}
	
\begin{document}

\section*{Рюкзак}

\fbox{\begin{minipage}{35em}
\textbf{Дано:}
\begin{itemize}
	\item веса $w[i] \text{, } i=\overline{1,n}$ 
	\item стоимтости $c[i] \text{, } i=\overline{1,n}$ 
	\item $W$ - максимальный вес, вместимость рюкзака
\end{itemize}

\textbf{Найдите подмножество элементов I такое, что:}
\begin{itemize}
	\item Общий вес предметов в $I$ не превышает $W$, $\sum_{i \in I} w[i] \leq W$
	\item Общая стоимость элементов в $I$ максимальна $W$, $max_{I \subset \{1, \dots, n\}} \sum_{i \in I} c[i] $
\end{itemize}
\end{minipage}}

\section*{Базовое решение для рюкзака}

\begin{itemize}
	\item[-] $dp[i][c]$ - сохраняет минимально достижимый вес для первого $i$ элементы со значениями $c$.
	\item[-] $0 \leq i \leq n$, $0 \leq c \leq C$ 
	\item[-] dp пересчет: \\
	\[
	dp[i][c] =
	\begin{cases} 
	dp[i-1][c], & \text{если } c[i] > c, \\
	\min\left(dp[i-1][c], dp[i-1][c - c[i]] + w[i]\right), & \text{иначе}.
	\end{cases}
	\]

\end{itemize}

\textbf{Итоговая ассимптотика: $\underline{O}(n * C_{max})$}

\section*{Аппроксимационный полиномиальный рюкзак}

\fbox{\begin{minipage}{35em}
\textbf{Given:}
\begin{itemize}
	\item веса $w[i] \text{, } i=\overline{1,n}$ 
	\item стоимости $c[i] \text{, } i=\overline{1,n}$ 
	\item $W$ - максимальный вес, вместимость рюкзака
	\item $\varepsilon$ - допустимая ошибка.
\end{itemize}

\textbf{Найдите подмножество элементов I такое, что:}
\begin{itemize}
	\item Общий вес предметов в $I$ не превышает $W$, $\sum_{i \in I} w[i] \leq W$
	\item Максимальная общая стоимость предметов в $I$ отличается от реального ответа не более чем на $1+\varepsilon$
	\item Алгоритм работает в полиномиальной асимптотике.
\end{itemize}
\end{minipage}}

\section*{Приближенное полиномиальное решение}

\begin{enumerate}
	\item Найдем $K = \frac{\varepsilon * C_{max}}{n}$, $C_{max} = max_{i=\overline{1,n}} c[i]$
	\item $c^{'}[i] = \lfloor{\frac{c[i]}{K}} \rfloor$
	\item Решите задачу о рюкзаке с помощью $c^{'}[i]$
\end{enumerate}

\textbf{окончательная асимптотика: $\underline{O}(n^2 \lfloor \frac{C_{max}}{K} \rfloor) = \underline{O}(n^2 \lfloor \frac{n}{\varepsilon} \rfloor)$}

\textbf{Корректность:} 
Пусть $S^{'}$ - множество предметов который мы взяли в работе полиномиального алгоритма. Докажем следующую лемму. \\
\textbf{Лемма} 
$C(S^{'}) \geq (1 -\varepsilon ) * OPT$, где $OPT$ - это правильный ответ, $C(S)$ - возвращает суммарную стоимость предметов из множества $S$: $C(S) = \sum_{i \in S} c[i]$
\begin{proof}
	Пусть $C^{'}(S)$ - это сумма стоимомтей измененных объектов: $C^{'}(S) = \sum_{i \in S} c^{'}[i]$

	Пусть $S$ - это оптимальный набор, то есть $C(S) = OPT$. После проведения преобразований с объектами, их стоимости будут разделены на $K$ и округлены в меншьшую сторону. Тогда для каждого отдельного объекта $a$ верно: $K*c^{'}[a] \leq c[a]$.

	Тогда для множества $S$ в целом, применяя это неравенство к каждому его объекту имеем: 
	$$ C(S) - K * C^{'}(S) \leq n * K \ (*)$$

	После решения задачи о рюкзаке для измененного набора данных мы получаем ответ множество $S^{'}$. После шага динамического программирования мы получаем набор, который является оптимальным для масштабируемого экземпляра
	и, следовательно, должен быть по крайней мере таким же хорошим, как выбор набора $S$ с меньшей прибылью:
	$$C(S^{'}) \geq K * C^{'}(S)\ (**)$$

	Далее из $(*),\ (**)$ получаем: $C(S^{'}) \geq K * C^{'}(S) \geq C(S) - n * K \geq OPT - \varepsilon * C_{max} \geq (1 - \varepsilon)*OPT$, так как $OPT \geq C_{max}$, если это не так то мы просто можем выкинуть объект максимальной стоимости из рассмотрения.
\end{proof}
% \url{https://math.mit.edu/~goemans/18434S06/knapsack-katherine.pdf}.


\end{document}
